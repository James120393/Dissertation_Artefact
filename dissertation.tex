%% LaTeX template for BSc Computing for Games final year project dissertations
%% by Edward Powley
%% Games Academy, Falmouth University, UK

%% Based on:
%% bare_jrnl.tex
%% V1.4b
%% 2015/08/26
%% by Michael Shell
%% see http://www.michaelshell.org/
%% for current contact information.
%%
%% This is a skeleton file demonstrating the use of IEEEtran.cls
%% (requires IEEEtran.cls version 1.8b or later) with an IEEE
%% journal paper.
%%
%% Support sites:
%% http://www.michaelshell.org/tex/ieeetran/
%% http://www.ctan.org/pkg/ieeetran
%% and
%% http://www.ieee.org/

%%*************************************************************************
%% Legal Notice:
%% This code is offered as-is without any warranty either expressed or
%% implied; without even the implied warranty of MERCHANTABILITY or
%% FITNESS FOR A PARTICULAR PURPOSE! 
%% User assumes all risk.
%% In no event shall the IEEE or any contributor to this code be liable for
%% any damages or losses, including, but not limited to, incidental,
%% consequential, or any other damages, resulting from the use or misuse
%% of any information contained here.
%%
%% All comments are the opinions of their respective authors and are not
%% necessarily endorsed by the IEEE.
%%
%% This work is distributed under the LaTeX Project Public License (LPPL)
%% ( http://www.latex-project.org/ ) version 1.3, and may be freely used,
%% distributed and modified. A copy of the LPPL, version 1.3, is included
%% in the base LaTeX documentation of all distributions of LaTeX released
%% 2003/12/01 or later.
%% Retain all contribution notices and credits.
%% ** Modified files should be clearly indicated as such, including  **
%% ** renaming them and changing author support contact information. **
%%*************************************************************************


\documentclass[journal]{IEEEtran}

\usepackage{graphicx}
% to embed the file `myreferences.bib` in your `.tex` file
% Insert additional usepackage commands here


\begin{document}
%
% paper title
% Titles are generally capitalized except for words such as a, an, and, as,
% at, but, by, for, in, nor, of, on, or, the, to and up, which are usually
% not capitalized unless they are the first or last word of the title.
% Linebreaks \\ can be used within to get better formatting as desired.
% Do not put math or special symbols in the title.
\title{Your title here}
%
%
% author name
\author{Your name here}

% The paper headers -- please do not change these, but uncomment one of them as appropriate
% Uncomment this one for COMP320
\markboth{COMP320: Research Review and Proposal}{COMP320: Research Review and Proposal}
% Uncomment this one for COMP360
% \markboth{COMP360: Dissertation}{COMP360: Dissertation}

% make the title area
\maketitle

% As a general rule, do not put math, special symbols or citations
% in the abstract or keywords.
\begin{abstract}
The abstract goes here.
\end{abstract}

\section{Introduction}
% The very first letter is a 2 line initial drop letter followed
% by the rest of the first word in caps.
% 
% form to use if the first word consists of a single letter:
% \IEEEPARstart{A}{demo} file is ....
% 
% form to use if you need the single drop letter followed by
% normal text (unknown if ever used by the IEEE):
% \IEEEPARstart{A}{}demo file is ....
% 
% Some journals put the first two words in caps:
% \IEEEPARstart{T}{his demo} file is ....
% 
% Here we have the typical use of a "T" for an initial drop letter
% and "HIS" in caps to complete the first word.
\IEEEPARstart{T}{his} demo file is intended to serve as a ``starter file''
for your final year project dissertation.
Use this template for your literature review and proposal in COMP320,
which you will then expand into your dissertation in COMP360.

Insert more sections here.
Use BibTeX~\cite{GoalDriven} \cite{MM} \cite{Fuzzy} to cite relevant literature. \cite{Blizzard}


\section{Conclusion}
The conclusion goes here.

% references section

\bibliographystyle{IEEEtran}
\bibliography{references}

% Appendices

\appendices
\section{First appendix}
Appendices are optional. Delete or comment out this part if you do not need them.

% that's all folks
\end{document}
